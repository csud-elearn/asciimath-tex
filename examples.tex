\documentclass{article}
\usepackage{amsmath}
\usepackage{ulem}
\usepackage{asciimth}

\def\startTrace{\message{Starting trace...@}%
    \tracingmacros=2\tracingassigns=2\tracingcommands=2%
}
\def\endTrace{\message{Ending trace...@}\tracingmacros=0\tracingassigns=0\tracingcommands=0}

% OK, note this doesn't really test spaces well.
\begingroup
\catcode32=13%
\gdef\changeSpaces{%
\catcode32=13%
\let =\ %
}
\endgroup
\def\changeSpacesBack{\catcode32=10}

\def\testRow{\hline\changeSpaces\testRowA}
\def\testRowA#1{%
    \changeSpacesBack
    \exampleRow{\sanitizeVerbose#1\end}{`#1`}%
}
    
\def\exampleRow#1#2{%
    {\tt\char18#1\char18}%
    %& #2\tabularnewline%
    & \mathcell{#2}\tabularnewline
}

% This is a little bit of a hack, but it was the easiest way I could
% figure out to get the math looking pretty inside of a table cell.
\def\mathcell#1{{
    \fboxrule=0pt
    \fbox{$\displaystyle #1$}%
    }%
}

% We can't just set the catcodes of ^ and _ because we want them as letters
% in the \tt but as super/subscripts in the `...`.
\def\sanitizeVerbose{\futurelet\nextTok\sanitizeVerboseA}
\def\sanitizeVerboseA#1{%
    \ifx#1\end\let\next=\relax
    \else\ifx#1^\def\next{\char94\sanitizeVerbose}%
    \else\ifx#1_\def\next{\char95\sanitizeVerbose}%
    % If it's a macro, unpack the name.
    \else\ifcat\noexpand#1\def\def\next{\string#1\sanitizeVerbose}%
    \else\def\next{#1\sanitizeVerbose}%
    \fi\fi\fi\fi
    \next
}

\def\rawExampleRow{\hline\changeSpaces\readFirstPart}
\def\readFirstPart#1{\changeSpacesBack\exampleRow{#1}}

\parskip=10pt

\newenvironment{examples}[1]
    {\renewcommand{\arraystretch}{1.5}
    \renewcommand{\tabcolsep}{0.2cm}
    \begin{tabular}{|p{6cm}|p{3cm}|}
        \multicolumn{2}{c}{\textbf{#1}}\\}
    {\hline\end{tabular}}

%%%%%%%%%%%%%%%%%%%%%%%%%%%%%%%%%%%%%

\begin{document}


This document displays a list of examples of equations typeset with the
{\tt asciimth} package.  It serves both as documentation and as a comprehensive
test of the package.

\begin{section}{Tables}

\begin{examples}{Simple commands}
\testRow{9+alpha}
\testRow{ (9)}
\testRow{9+(4-(alpha))}
\testRow{gamma>2 implies gamma*gamma>4}
\testRow{x ge 0 implies x x ge 0}
\testRow{t}
\testRow{1-a}
\testRow{[2,3)}
\testRow{1 234}
\testRow{a b - a\,b}
\rawExampleRow{}{``}
\rawExampleRow{ }{` `}
\end{examples}

\begin{examples}{Fractions}
\testRow{9/2}
\testRow{(1//2]/(7//4)}
\testRow{(9)/(7)}
\testRow{ 2/3 + ((alpha+2)/2 * 5)/7}
\testRow{2/3/3}
\testRow{-5/zeta}
\testRow{z/((q*(2/beta to gamma)))}
\testRow{([2/3])/7}
\testRow{Phi = 1+1/(1+1/(1+cdots))}
\end{examples}

\begin{examples}{Fonts}
\testRow{hat(a)}
\testRow{x in bb(Z)}
\end{examples}


\begin{examples}{Exponents and subscripts}
\testRow{5^14}
\testRow{5_14}
\testRow{(9^-74)}
\testRow{ 125^-74}
\testRow{1/125^74}
\testRow{(1/125)^74}
\testRow{alpha^(2+3)/5}
\testRow{x^n y_1^-t alpha^-(beta-z)}
\testRow{(q r s)_(123)^-[456/z]}
\testRow{a_1^2}
\testRow{(a+b)_(k_1)^(x+2)}
\testRow{sum_(n=1)^infty 1/n^2}
\testRow{int_pi^infty 1/(sin x)^2\,d x}
\end{examples}

\begin{examples}{Square roots}
\testRow{sqrt 2}
\testRow{sqrt(2+4/5)}
\testRow{2/sqrt(3alpha)}
\testRow{sqrt sqrt 3}
\testRow{(2+3)/sqrt sqrt sqrt 3}
\testRow{root n (x^2+1)}
\testRow{(-b pm sqrt(b^2-4a c))/(2a)}
\end{examples}

\end{section}

\begin{section}{Example usage}

Finally, as an example of the various ways in which you can use this package, we
present a formula for the root of a cubic polynomial (via Wikipedia):

When `x^3+a x^2+b x+c`, one of the roots is
\[`
-1/3(a
+root3((m+sqrt(m^2-4k^3))/2)
+root3((m-sqrt(m^2-4k^3))/2))
`\]
where:
\begin{align*}
m&=`2a^3-9a`,\\
k&=`a^2-3b`.\\
\end{align*}

The above text was typeset via:
\begin{verbatim}
When `x^3+a x^2+b x+c`, one of the roots is
\[`
-1/3(a
+root3((m+sqrt(m^2-4k^3))/2)
+root3((m-sqrt(m^2-4k^3))/2))
`\]
where:
\begin{align*}
m&=`2a^3-9a`,\\
k&=`a^2-3b`,\\
\end{align*}
\end{verbatim}

\end{section}


\begin{section}{Current issues}

Currently there's some typesetting issues caused by the fact that we always use {\tt
\string\left} and {\tt\string\right} delimiters.  However, I expect those can be
fixed by using the \texttt{nath} package.  This may require \sout{stealing} adapting
some of its macros for my own use, though, since that package isn't compatible with
\texttt{asciimath}.
\begin{itemize}
\item In the first row, the spacing between `f` and the parenthesis is too large;
compare $f(x)$ versus `f(x)`.
\item The parenthesis around the inner `sum` is too large; they erroneously try
to wrap the subscript `j`.
\end{itemize}

\begin{examples}{Parenthesis tests}
\testRow{f(x)+g(x/y)}
\testRow{(x+y)(x-y)}
\testRow{f(x^2)+f(2^(2/y))}
\testRow{sum(x_i)+sum(x/y)}
\testRow{sum_i(sum_j(2*x))}
\end{examples}

\end{section}


\end{document}

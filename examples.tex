\documentclass{article}
\usepackage{amsmath}
\usepackage{ulem}
\usepackage{asciimth}

\def\startTrace{\message{Starting trace...$}%
    \tracingmacros=2\tracingassigns=2\tracingcommands=2%
}
\def\endTrace{\message{Ending trace...$}\tracingmacros=0\tracingassigns=0\tracingcommands=0}

% OK, note this doesn't really test spaces well.
\begingroup
\catcode32=13%
\gdef\changeSpaces{%
\catcode32=13%
\let =\ %
}
\endgroup
\def\changeSpacesBack{\catcode32=10}

\def\testRow{\hline\changeSpaces\testRowA}
\def\testRowA#1{%
    \changeSpacesBack
    \exampleRow{\sanitizeVerbose#1\end}{`#1`}%
}

\def\exampleRow#1#2{%
    {\tt\char18#1\char18}%
    %& #2\tabularnewline%
    & \mathcell{#2}\tabularnewline
}

% This is a little bit of a hack, but it was the easiest way I could
% figure out to get the math looking pretty inside of a table cell.
\def\mathcell#1{{
    \fboxrule=0pt
    \fbox{$\displaystyle #1$}%
    }%
}

% We can't just set the catcodes of ^ and _ because we want them as letters
% in the \tt but as super/subscripts in the `...`.
\def\sanitizeVerbose{\futurelet\nextTok\sanitizeVerboseA}
\def\sanitizeVerboseA{%
    \ifx\nextTok\bgroup\let\next=\sanitizeGroupAndRepeat
    \else\let\next=\sanitizeVerboseB
    \fi\next
}
\def\sanitizeGroupAndRepeat#1{%
    \{\sanitizeVerbose#1\end\}\sanitizeVerbose
}
\def\sanitizeVerboseB#1{%
    \ifx#1\end\let\next=\relax
    \else\ifx#1^\def\next{\char94\sanitizeVerbose}%
    \else\ifx#1_\def\next{\char95\sanitizeVerbose}%
    % If it's a macro, unpack the name.
    \else\ifcat\noexpand#1\def\def\next{\string#1\sanitizeVerbose}%
    \else\def\next{#1\sanitizeVerbose}%
    \fi\fi\fi\fi
    \next
}

\def\rawExampleRow{\hline\changeSpaces\readFirstPart}
\def\readFirstPart#1{\changeSpacesBack\exampleRow{#1}}

\newenvironment{examples}[1]
    {\renewcommand{\arraystretch}{1.5}
    \renewcommand{\tabcolsep}{0.2cm}
    \begin{tabular}{|p{6cm}|p{3.5cm}|}
        \multicolumn{2}{c}{\textbf{#1}}\\}
    {\hline\end{tabular}}

%%%%%%%%%%%%%%%%%%%%%%%%%%%%%%%%%%%%%

\title{The {\tt asciimth} package}
\author{Judah Jacobson}
\begin{document}
\maketitle

\section{Introduction}
The {\tt asciimth} package provides a more readable way to typeset
mathematics in \LaTeX{}.  The interface is very simple: any text surrounded by
left-quotes (\char18\ldots\char18) will be parsed by the {\tt asciimth}
engine.  For example, typing {\tt\char18 (x-1)\char94(2x)/(4-3)\char18} in
the middle of a paragraph produces `(x-1)^(2x)/(4-3)`.  

That notation can be mixed with standard \TeX{} math commands by
inserting a left-quoted expression in the middle of an existing expression or
equation.  For example, \verb|$\frac{1}{2}-`3/4`$| is typeset as 
$\frac{1}{2}-`3/4`$.  It can also be embedded into display math
(\verb|$$| or \verb|\[|) or a math environment such as {\tt multiline} or
\texttt{align}.

For compatibility with the common use of left-quotes, the empty expression 
{\tt\char18\char18} is typeset as the begin-quotes symbol (``).

This document displays a range of example equations as typeset by {\tt
asciimth}.  It serves both as documentation and as a comprehensive
test of the package.

\begin{section}{Tables}

\begin{examples}{Simple commands}
\testRow{9+alpha}
\testRow{gamma>2 implies gamma*gamma>4}
\testRow{x ge 0 implies x x ge 0}
\testRow{t}
\testRow{1-a}
\testRow{1 234}
\testRow{a b - a\,b}
\testRow{ (9)}
\testRow{9+(4-(alpha))}
\testRow{(-)}
\testRow{[2,3)}
\rawExampleRow{ }{` `}
\rawExampleRow{}{``}
\end{examples}

\begin{examples}{Fractions and parentheses}
\testRow{9/2}
\testRow{(1//2]/(7//4)}
\testRow{(9)/(7)}
\testRow{ 2/3 + ((alpha+2)/2 * 5)/7}
\testRow{2/3/3}
\testRow{-5/zeta}
\testRow{z/((q*(2/beta to gamma)))}
\testRow{([2/3])/7}
\testRow{Phi = 1+1/(1+1/(1+cdots))}
\testRow{left|2/3right|}
\testRow{(2/3 middle| 7)}
\end{examples}

\begin{examples}{Fonts}
\testRow{hat(a)}
\testRow{x in bb(Z)}
\end{examples}


\begin{examples}{Exponents and subscripts}
\testRow{5^14}
\testRow{5_14}
\testRow{(9^-74)}
\testRow{ 125^-74}
\testRow{1/125^74}
\testRow{(1/125)^74}
\testRow{alpha^(2+3)/5}
\testRow{x^n y_1^-t alpha^-(beta-z)}
\testRow{(q r s)_(123)^-[456/z]}
\testRow{a_1^2}
\testRow{(x)_-i + q^r_(s)}
\testRow{(a+b)_(k_1)^(x+2)}
\testRow{sum_(n=1)^infty 1/n^2}
\testRow{int_pi^infty 1/(sin x)^2\,d x}
\end{examples}

\begin{examples}{Square roots}
\testRow{sqrt 2}
\testRow{sqrt(2+4/5)}
\testRow{2/sqrt(3alpha)}
\testRow{sqrt sqrt 3}
\testRow{(2+3)/sqrt sqrt sqrt 3}
\testRow{root n (x^2+1)}
\testRow{(-b pm sqrt(b^2-4a c))/(2a)}
\end{examples}


\par
\newcommand\norm[1]{\left|\left|#1\right|\right|}
In the next example, assume that we've defined:
\begin{verbatim}
\newcommand\norm[1]{\left|\left|#1\right|\right|}
\end{verbatim}
\begin{examples}{Braces and user-defined macros}
\testRow{norm {x}}
\testRow{norm{(x)}}
\testRow{norm { x-y}^2+z}
\testRow{t*{2+alpha}^2}
\testRow{t*{(2+alpha)}^2}
\testRow{ {2+x}/{gamma^2beta}}
\testRow{ \{2/(x+y)\}^3}
\end{examples}

\end{section}

\begin{section}{Example usage}

Finally, as an example of the various ways in which you can use this package, we
present a formula for the root of a cubic polynomial (via Wikipedia):

When `x^3+a x^2+b x+c`, one of the roots is
\[`
-1/3(a
+root3((m+sqrt(m^2-4k^3))/2)
+root3((m-sqrt(m^2-4k^3))/2))
`\]
where:
\begin{align*}
m&=`2a^3-9a`,\\
k&=`a^2-3b`.\\
\end{align*}

The above text was typeset via:
\begin{verbatim}
When `x^3+a x^2+b x+c`, one of the roots is
\[`
-1/3(a
+root3((m+sqrt(m^2-4k^3))/2)
+root3((m-sqrt(m^2-4k^3))/2))
`\]
where:
\begin{align*}
m&=`2a^3-9a`,\\
k&=`a^2-3b`,\\
\end{align*}
\end{verbatim}

\end{section}


\begin{section}{Layout tests}

These examples cover a few issues which are caused by the fact that we always
use {\tt\string\left} and {\tt\string\right} delimiters:
\begin{itemize}
\item Using {\tt\string\left(\ldots\string\right)} causes different
spacing than
{\tt(\ldots)}.  That issue has been resolved.
\item The parentheses around the inner `sum` in the last example
are too large; they unnecessarily grow
to cover the subscript `j`.
\end{itemize}

\begin{examples}{Parenthesis tests}
\testRow{f(x)+g(x/y)}
\testRow{(x+y)(x-y)}
\testRow{f(x^2)+f(2^(2/y))}
\testRow{(x^2)^2}
\testRow{sum(x_i)+sum(x/y).}
\testRow{sum_i(sum_j(2*x))}
\end{examples}

\end{section}


\end{document}
